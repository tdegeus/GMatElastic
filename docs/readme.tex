\documentclass[namecite, fleqn]{goose-article}

\title{Linear elasticity}

\author{Tom W.J.\ de Geus}

\hypersetup{pdfauthor={T.W.J. de Geus}}

\begin{document}

\maketitle

\section{Constitutive model}

The stress, $\bm{\sigma}$, is set by to the strain, $\bm{\varepsilon}$,
through the following linear relation:
\begin{equation}
    \bm{\sigma}
    \equiv K \mathrm{tr}\left( \bm{\varepsilon} \right)
    + 2 G \bm{\varepsilon}_\mathrm{d}
    \equiv \mathbb{C} : \bm{\varepsilon}
\end{equation}
wherein $\mathbb{C}$ is the elastic stiffness, which reads:
\begin{align}
    \mathbb{C}
    &\equiv K \bm{I} \otimes \bm{I}
    + 2 G (\mathbb{I}_\mathrm{s} - \tfrac{1}{3} \bm{I} \otimes \bm{I} )
    \\
    &= K \bm{I} \otimes \bm{I}
    + 2 G \, \mathbb{I}_\mathrm{d}
\end{align}
with $K$ and $G$ the bulk and shear modulus respectively.
See \cref{sec:ap:nomenclature} for nomenclature, including definitions of the unit tensors.

\section{Consistency check}

To check if the derived tangent $\mathbb{C}$ a \emph{consistency check} can be performed.
A (random) perturbation $\delta \bm{\varepsilon}$ is applied.
The residual is compared to that predicted by the tangent.
For the general case of linearisation, the following holds:
\begin{equation}
    \bm{\sigma}\big( \bm{\varepsilon}_\star + \delta \bm{\varepsilon} \big)
    = \bm{\sigma}\big( \bm{\varepsilon}_\star \big)
    + \mathbb{C} \big( \bm{\varepsilon}_\star \big) : \delta \bm{\varepsilon}
    + \mathcal{O}(\delta \bm{\varepsilon}^2)
\end{equation}
or
\begin{equation}
    \underbrace{
        \bm{\sigma}\big( \bm{\varepsilon}_\star + \delta \bm{\varepsilon} \big)
        - \bm{\sigma}\big( \bm{\varepsilon}_\star \big)
    }_{
        \displaystyle \delta \bm{\sigma}
    }
    - \mathbb{C} \big( \bm{\varepsilon}_\star \big) : \delta \bm{\varepsilon}
    = \mathcal{O}(\delta \bm{\varepsilon}^2)
\end{equation}
This allows the introduction of a relative error
\begin{equation}
    \eta =
    \Big|\Big|
        \delta \bm{\sigma}
        - \mathbb{C}(\bm{\varepsilon}_\star) : \delta \bm{\varepsilon}
    \Big|\Big|
    /
    \Big|\Big| \delta \bm{\sigma} \Big|\Big|
\end{equation}
This \emph{truncation error} thus scales as $\eta \sim || \delta \bm{\varepsilon} ||^2$
as depicted in \cref{fig:consistency:expected}.
As soon as the error becomes sufficiently small the numerical \emph{rounding error}
becomes more dominant, the scaling thereof is also included in \cref{fig:consistency:expected}.

\begin{figure}[htp]
    \centering
    \includegraphics[width=.5\textwidth]{figures/consistency}
    \caption{Expected behaviour of the consistency check, see \citet[p.~9]{Heath2002}.}
    \label{fig:consistency:expected}
\end{figure}

Because this model is linear there is no truncation error, the measurement of $\eta$ and a
function of $|| \delta \bm{\varepsilon} ||$ thus only displays a rounding error,
as depicted in \cref{fig:consistency}.

\begin{figure}[htp]
    \centering
    \includegraphics[width=.5\textwidth]{examples/consistency}
    \caption{Measured consistency check, cf.\ \cref{fig:consistency:expected}.}
    \label{fig:consistency}
\end{figure}

\appendix

\section{Nomenclature}
\label{sec:ap:nomenclature}

\paragraph{Tensor products}
\vspace*{.5eM}

\begin{itemize}

    \item Dyadic tensor product
    \begin{align}
        \mathbb{C} &= \bm{A} \otimes \bm{B} \\
        C_{ijkl} &= A_{ij} \, B_{kl}
    \end{align}

    \item Double tensor contraction
    \begin{align}
        C &= \bm{A} : \bm{B} \\
        &= A_{ij} \, B_{ji}
    \end{align}

\end{itemize}

\paragraph{Tensor decomposition}
\vspace*{.5eM}

\begin{itemize}

    \item Deviatoric part $\bm{A}_\mathrm{d}$ of an arbitrary tensor $\bm{A}$:
    \begin{equation}
        \mathrm{tr}\left( \bm{A}_\mathrm{d} \right) \equiv 0
    \end{equation}
    and thus
    \begin{equation}
        \bm{A}_\mathrm{d} = \bm{A} - \tfrac{1}{3} \mathrm{tr}\left( \bm{A} \right)
    \end{equation}

\end{itemize}

\paragraph{Fourth order unit tensors}
\vspace*{.5eM}

\begin{itemize}

    \item Unit tensor:
    \begin{equation}
        \bm{A} \equiv \mathbb{I} : \bm{A}
    \end{equation}
    and thus
    \begin{equation}
        \mathbb{I} = \delta_{il} \delta{jk}
    \end{equation}

    \item Right-transposition tensor:
    \begin{equation}
        \bm{A}^T \equiv \mathbb{I}^{RT} : \bm{A} = \bm{A} : \mathbb{I}^{RT}
    \end{equation}
    and thus
    \begin{equation}
      \mathbb{I}^{RT} = \delta_{ik} \delta_{jl}
    \end{equation}

    \item Symmetrisation tensor:
    \begin{equation}
        \mathrm{sym} \left( \bm{A} \right) \equiv \mathbb{I}_\mathrm{s} : \bm{A}
    \end{equation}
    whereby
    \begin{equation}
        \mathbb{I}_\mathrm{s} = \tfrac{1}{2} \left( \mathbb{I} + \mathbb{I}^{RT} \right)
    \end{equation}
    This follows from the following derivation:
    \begin{align}
        \mathrm{sym} \left( \bm{A} \right) &= \tfrac{1}{2} \left( \bm{A} + \bm{A}^T \right)
        \\
        &= \tfrac{1}{2} \left( \mathbb{I} : \bm{A} + \mathbb{I}^{RT} : \bm{A} \right)
        \\
        &= \tfrac{1}{2} \left( \mathbb{I} + \mathbb{I}^{RT} \right) : \bm{A}
        \\
        &= \mathbb{I}_\mathrm{s} : \bm{A}
    \end{align}

    \item Deviatoric and symmetric projection tensor
    \begin{equation}
        \mathrm{dev} \left( \mathrm{sym} \left( \bm{A} \right) \right)
        \equiv \mathbb{I}_\mathrm{d} : \bm{A}
    \end{equation}
    from which it follows that:
    \begin{equation}
        \mathbb{I}_\mathrm{d}
        = \mathbb{I}_\mathrm{s} - \tfrac{1}{3} \bm{I} \otimes \bm{I}
    \end{equation}

\end{itemize}

\bibliography{library}

\end{document}
